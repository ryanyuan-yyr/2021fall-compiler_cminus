\documentclass{beamer}

% for annotated equations %%%%%%%%%%%%%%%%
% \usepackage[dvipsnames]{xcolor}

\usepackage{tikz}
\usetikzlibrary{backgrounds}
\usetikzlibrary{arrows,shapes}
\usetikzlibrary{tikzmark}
\usetikzlibrary{calc}

\usepackage{amsmath}
\usepackage{amsthm}
\usepackage{amssymb}
\usepackage{mathtools, nccmath}
\usepackage{wrapfig}
\usepackage{comment}

% To generate dummy text
\usepackage{blindtext}


%color
%\usepackage[dvipsnames]{xcolor}
% \usepackage{xcolor}


%\usepackage[pdftex]{graphicx}
\usepackage{graphicx}
% declare the path(s) for graphic files
%\graphicspath{{../Figures/}}

% extensions so you won't have to specify these with
% every instance of \includegraphics
% \DeclareGraphicsExtensions{.pdf,.jpeg,.png}

% for custom commands
\usepackage{xspace}

% table alignment
\usepackage{array}
\usepackage{ragged2e}
\newcolumntype{P}[1]{>{\RaggedRight\hspace{0pt}}p{#1}}
\newcolumntype{X}[1]{>{\RaggedRight\hspace*{0pt}}p{#1}}

% color box
\usepackage{tcolorbox}


% for tikz
\usepackage{tikz}
%\usetikzlibrary{trees}
\usetikzlibrary{arrows,shapes,positioning,shadows,trees,mindmap}
% \usepackage{forest}
\usepackage[edges]{forest}
\usetikzlibrary{arrows.meta}
\colorlet{linecol}{black!75}
\usepackage{xkcdcolors} % xkcd colors


% for colorful equation
\usepackage{tikz}
\usetikzlibrary{backgrounds}
\usetikzlibrary{arrows,shapes}
\usetikzlibrary{tikzmark}
\usetikzlibrary{calc}
% Commands for Highlighting text -- non tikz method
\newcommand{\highlight}[2]{\colorbox{#1!17}{$\displaystyle #2$}}
%\newcommand{\highlight}[2]{\colorbox{#1!17}{$#2$}}
\newcommand{\highlightdark}[2]{\colorbox{#1!47}{$\displaystyle #2$}}

% my custom colors for shading
% \colorlet{mhpurple}{Plum!80}


% Commands for Highlighting text -- non tikz method
\renewcommand{\highlight}[2]{\colorbox{#1!17}{#2}}
\renewcommand{\highlightdark}[2]{\colorbox{#1!47}{#2}}

%%%%%%%%%%%%%%%%%%%%%%%%%%%%%%%%%%%%

\usepackage{listings}
\usepackage{qtree}
\usepackage{multirow}

\usetheme {Copenhagen}

% Title page details
\title{Compilers Principles Lab5 Report }
\subtitle{\it{Language Feature Extension}}
\author{Yurun Yuan}
\institute{CS, USTC}
\date{\today}

\begin{document}

\lstset{language=C,keywordstyle={\bfseries \color{blue}}}

\begin{frame}
    \titlepage
\end{frame}

\begin{frame}{Outline}
    \tableofcontents
\end{frame}

\AtBeginSection[]
{
    \begin{frame}{Outline}
        \tableofcontents[currentsection]
    \end{frame}
}

\AtEndDocument{
    \begin{frame}
        \centering \Large
        \emph{Thank You.}
    \end{frame}
}

\section{Type System Extension}

\begin{frame}{Declarations of Variables}
    \begin{columns}
        \begin{column}[]{0.5\textwidth}
            \begin{itemize}
                \item Pointers

                      \lstinline{int *ptr;}

                \item Function

                      \lstinline{int func(int);}

                \item Arrays

                      \lstinline{int array[42];}
            \end{itemize}
        \end{column}
        \begin{column}[]{0.5\textwidth}
            \begin{itemize}
                \item Structures

                      \lstinline{struct S\{...\};}

                      \lstinline{struct S s;}

                      \lstinline{struct S\{...\} s;}

                      \lstinline{struct\{...\} anonymous;}
            \end{itemize}
        \end{column}
    \end{columns}
\end{frame}

\subsection{Pointers and Arrays}

\begin{frame}
    \frametitle{Declaration Grammar}
    \begin{block}{Declaration grammar}
        $$
            \begin{aligned}
                \text{declaration} & \rightarrow & \text{type-specifier declarator ;}
            \end{aligned}
        $$
    \end{block}

    \only<1>{
        % \begin{block}{Type specifiers}
        %     $$
        %         \begin{aligned}
        %             \text{type-specifier} & \rightarrow & \text{\textbf{int} \textbar
        %                 \textbf{float}\textbar\textbf{void}}
        %         \end{aligned}
        %     $$
        % \end{block}

        \begin{example}
            % \begin{center}
            %     \begin{tabular}{l |l | l }
            %         Variable declaration       & Type specifier        & Declarator \\
            %         \hline \hline
            %         \lstinline{int *ptr;}       & \lstinline{int} & \lstinline{*ptr}                   \\
            %         \lstinline{int array[42];}     & \lstinline{int} & \lstinline{array[42]}
            %     \end{tabular}
            % \end{center}
            \
            \\


            \begin{equation*}
                \tikzmarknode{ts}{\highlight{red}{\lstinline{int}}}\ \tikzmarknode{d}{\highlight{blue}{\lstinline{* a[2]}}} \lstinline{;}
            \end{equation*}
            \begin{tikzpicture}[overlay,remember picture,>=stealth,nodes={align=left,inner ysep=1pt},<-]
                % For "X"
                \path (ts.north) ++ (0,0.5em) node[anchor=south east,color=red!67] (scalep){\textbf{type-specifier}};
                \draw [color=red!87](ts.north) |- ([xshift=-0.3ex,color=red]scalep.south west);
                % For "S"
                \path (d.south) ++ (0,-0.5em) node[anchor=north west,color=blue!67] (mean){\textbf{declarator}};
                \draw [color=blue!57](d.south) |- ([xshift=-0.3ex,color=blue]mean.south east);
            \end{tikzpicture}
        \end{example}
    }

    \only<2>{
        \begin{itemize}
            \item Pointer declarators

                  \lstinline{int}\ \highlight{blue}{\lstinline{* ptr}} \lstinline{;}
            \item Function declarators

                  \lstinline{int}\ \highlight{blue}{\lstinline{func(float)}} \lstinline{;}
            \item Array declarators

                  \lstinline{int}\ \highlight{blue}{\lstinline{array[42]}} \lstinline{;}
        \end{itemize}
    }
\end{frame}

\begin{frame}
    \frametitle{Crux: declarators can get mixed up}

    \begin{example}
        \lstinline{int (* a[2])(int);}
        \\
        \begin{center}
            \begin{tabular}{l |l}
                Nested declarators & Type expression                     \\
                \hline \hline
                \lstinline{(* a[2])(int)}    & \lstinline{int}                              \\
                \lstinline{* a[2]}   & \lstinline{int}$\rightarrow$\lstinline{int}  \\
                \lstinline{a[2]}   & pointer(\lstinline{int}$\rightarrow$\lstinline{int}) \\
                \lstinline{a}    & array(2, pointer(\lstinline{int}$\rightarrow$\lstinline{int}))
            \end{tabular}
        \end{center}
    \end{example}
\end{frame}

\begin{frame}
    \frametitle{Trial 1: a naive solution}
    \begin{example}
        \lstinline{int (* a[2])(int);}
    \end{example}
    \only<1>{$$
            \begin{array}{rlll}
                \text{declarator} & \rightarrow & \text{* declarator}                & \textbf{\textit{Pointer declarator}}  \\
                                  & |           & \text{declarator ( type, ... )}    & \textbf{\textit{Funtcion declarator}} \\
                                  & |           & \text{declarator [ \textbf{int} ]} & \textbf{\textit{Array declarat}}      \\
                                  & |           & \text{( declarator )}              & \                                     \\
                                  & |           & \textbf{ID}                        & \
            \end{array}
        $$}

    \only<2>{
        \begin{block}{Problem: ambiguity}
            \begin{columns}
                \begin{column}[]{0.5\textwidth}
                    \Tree [.{\lstinline{(* a[2])(int)}} [.{\alert{\lstinline{(* a[2])}}}
                            {\lstinline{*}}
                            [.{\lstinline{a[2]}}
                                    {\lstinline{a}} {\lstinline{[2]}}
                            ]
                    ] {\lstinline{(int)}} ]



                \end{column}
                \begin{column}[]{0.5\textwidth}
                    \Tree [.{\lstinline{(* a[2])(int)}} [.{\alert{\lstinline{(* a[2])}}}
                            [.{\lstinline{* a}}
                                    {\lstinline{*}}
                                    {\lstinline{a}}
                            ]
                            {\lstinline{[2]}}
                    ] {\lstinline{(int)}} ]
                \end{column}
            \end{columns}
        \end{block}
    }

\end{frame}

\begin{frame}
    \frametitle{Trial 2: eliminating ambiguity}
    \begin{columns}
        \only<1>{
            \begin{column}[]{0.4\textwidth}
                \begin{tabular}{l |l}
                    \hline
                    Preced             & Operation                  \\
                    \hline \hline
                    0                  & Parenthesis \lstinline{()} \\
                    \hline
                    \multirow{2}{*}{1} & Call \lstinline{()}        \\
                    ~                  & subscript\lstinline{[]}    \\
                    \hline
                    2                  & Dereference \lstinline{*}  \\
                    \hline
                \end{tabular}
            \end{column}}

        \only<2>{
            \begin{column}[]{0.5\textwidth}
                \begin{example}
                    \lstinline{int} \highlight{blue}{\lstinline{* array[2]}}\lstinline{;}
                    \\ \hspace*{\fill} \\

                    \Tree[.{decl}
                        *
                            [.{decl}
                                    [.{decl-1}
                                            [.{decl-1}
                                                    [.{factor}
                                                            {\textbf{ID}}
                                                    ]
                                            ]
                                            {\lstinline{\[int\]}}
                                    ]
                            ]
                    ]

                    % \Tree [.{\lstinline{(* a[2])(int)}} [.{\lstinline{(* a[2])}}
                    %         {\lstinline{*}}
                    %         [.{\lstinline{a[2]}}
                    %                 {\lstinline{a}} {\lstinline{[2]}}
                    %         ]
                    % ] {\lstinline{(int)}} ]
                \end{example}
            \end{column}
        }
        \begin{column}[]{0.6\textwidth}
            \begin{itemize}
                \small
                \item Parentheses and identifiers
                      \begin{eqnarray*}
                          \text{factor} & \rightarrow & \textbf{ID} \\
                          & | & ( \text{ decl } )
                      \end{eqnarray*}
                \item Function calls and array subscripting
                      \begin{eqnarray*}
                          \text{decl-1} & \rightarrow & \text{decl-1}(\text{type, ...}) \\
                          & | & \text{decl-1}[\textbf{int}]\\
                          & | &\text{factor}
                      \end{eqnarray*}
                \item Dereference
                      \begin{eqnarray*}
                          \text{decl} & \rightarrow & \textbf{*}\text{ decl} \\
                          & | & \text{decl-1}
                      \end{eqnarray*}
            \end{itemize}
        \end{column}
    \end{columns}

\end{frame}

\subsection{Structures}

\begin{frame}
    \frametitle{Declarations of Structures}

    \begin{itemize}
        \item Structure definition

              \lstinline{struct S\{...\};}

        \item Variable definition

              \lstinline{struct S s;}

        \item Structure and variable definition

              \lstinline{struct S\{...\} s;}

        \item Anonymous structure

              \lstinline{struct\{...\} anonymous;}
    \end{itemize}
\end{frame}

\begin{frame}
    \frametitle{Extension of the declaration grammar}

    \begin{block}{Declaration grammar}
        $$
            \begin{aligned}
                \text{declaration} & \rightarrow & \text{type-specifier declarator ;}
            \end{aligned}
        $$
    \end{block}

    \only<1>{
        \begin{center}
            \begin{tabular}{l |l | l }
                Declaration                                                                               & Type specifier                & Declarator            \\
                \hline \hline
                \highlight{red}{\lstinline{struct S\{...\}}}\lstinline{;}           & \lstinline{struct S\{...\}}     & $\emptyset$                                           \\
                \highlight{red}{\lstinline{struct S}} \highlight{blue}{\lstinline{s}}\lstinline{;}                    & \lstinline{struct S}         & \lstinline{s}                             \\
                \highlight{red}{\lstinline{struct S\{...\}}} \highlight{blue}{\lstinline{s}}\lstinline{;} & \lstinline{struct S\{...\}} & \lstinline{s}         \\
                \highlight{red}{\lstinline{struct\{...\}}} \highlight{blue}{\lstinline{anonymous}}\lstinline{;} & \lstinline{struct\{...\}} & \lstinline{anonymous}
            \end{tabular}
        \end{center}
    }

    \only<2>{
        $$
            \begin{array}{rll}
                \text{type-specifier}    & \rightarrow & \cdots                                                        \\
                                         & |           & \text{struct-definition}                                      \\
                                         & |           & \textbf{struct ID}                                            \\
                \\
                \text{struct-definition} & \rightarrow & \textbf{struct ID} \textbf{\{} \text{definitions} \textbf{\}} \\
                                         & |           & \textbf{struct} \textbf{\{} \text{definitions} \textbf{\}}    \\
                \\
                \text{declarator}        & \rightarrow & \cdots                                                        \\
                                         & |           & \epsilon
            \end{array}
        $$
    }

\end{frame}

\begin{frame}[fragile]
    \frametitle{Self reference}

    \begin{example}
        A naive implementation of the node of a forwarding linked list
        \begin{lstlisting}[language=C]
struct list_node{
    int elem;
    struct list_node* next;
};
        \end{lstlisting}
    \end{example}

    \only<1>{The identifier \fbox{\lstinline{struct list_node}} is visible within the definition of the structure, although \fbox{\lstinline{struct list_node}} is an incomplete type. }
    \only<2>{To render the identifier \fbox{\lstinline{struct list_node}} available, construct an empty structure before parsing the body of the structure definition, and complete the type after the definition parsing is done. }
\end{frame}

% \subsection{Initialization}

% \begin{frame}
%     \frametitle{Copy construction in declaration}

%     \begin{block}{Declaration grammar}
%         $$
%             \text{declaration}\rightarrow \text{type-specifier declarator} \textbf{ = } \text{expression}
%         $$
%     \end{block}

%     A declaration with initialization is transformed into a declaration with no initialization value and an Assignment.

% \end{frame}

\section{Operator Extension}

\begin{frame}
    \frametitle{Supported operators}

    \begin{columns}
        \begin{column}[]{0.5\textwidth}
            \begin{itemize}
                \item Arithmatic operators

                      \lstinline{+,-,*,/}
                \item Relational operators

                      \lstinline{>,<,>=,<=,==,!=}
                \item Assignment operator

                      \lstinline{=}
                \item Array subscript

                      \lstinline{array\[index\]}
            \end{itemize}
        \end{column}

        \begin{column}[]{0.5\textwidth}
            \begin{itemize}
                \item Pointer dereference

                      \lstinline{* ptr}
                \item Address of

                      \lstinline{\& lvalue}
                \item Member access

                      \lstinline{var.member}
                \item Function call

                      \lstinline{callable(params)}
            \end{itemize}
        \end{column}
    \end{columns}

\end{frame}

\subsection{Context Free Grammar of Expressions}

\begin{frame}
    \frametitle{Precedence and CFG}

    \only<1>{
        \begin{enumerate}
            \setcounter{enumi}{-1}
            \item Parenthesis $()$
                  \begin{eqnarray*}
                      \text{expr-0} & \rightarrow & \text{\textbf{ID}}\\
                      & | & \text{\textbf{Integer-literal}} \hspace{8em}\\
                      & | & \text{\textbf{Float-literal}}\\
                      & | & \text{( expr )}
                  \end{eqnarray*}

            \item Function call, array subscript, member access
                  \begin{eqnarray*}
                      \text{expr-1} & \rightarrow & \text{expr-1 ( args )}\\
                      & | & \text{expr-1 [ expr ]} \hspace{3em}\\
                      & | & \text{expr-1 \textbf{.} \textbf{ID}}\\
                      & | & \text{expr-0}
                  \end{eqnarray*}
        \end{enumerate}
    }

    \only<2>{
        \begin{enumerate}
            \setcounter{enumi}{1}
            \item Dereference, address of
                  \begin{eqnarray*}
                      \text{expr-2} & \rightarrow & \text{\textbf{*} expr-2}\\
                      & | & \text{\textbf{\&} expr-2} \hspace{8em}\\
                      & | & \text{expr-1}
                  \end{eqnarray*}

            \item Multiplication, division
                  \begin{eqnarray*}
                      \text{expr-3} & \rightarrow & \text{expr-3 \textbf{MulOp} expr-2}\\
                      & | & \text{expr-2}
                  \end{eqnarray*}

            \item Addition subtraction
                  \begin{eqnarray*}
                      \text{expr-4} & \rightarrow & \text{expr-4 \textbf{AddOp} expr-3}\\
                      & | & \text{expr-3}
                  \end{eqnarray*}
        \end{enumerate}
    }

    \only<3>{
        \begin{enumerate}
            \setcounter{enumi}{4}
            \item Relational operations
                  \begin{eqnarray*}
                      \text{expr-5} & \rightarrow & \text{expr-5 \textbf{RelOp} expr-4} \\
                      & | & \text{expr-4}
                  \end{eqnarray*}

            \item Assignment
                  \begin{eqnarray*}
                      \text{expr} & \rightarrow & \text{expr-5 \textbf{=} expr} \\
                      & | & \text{expr-5}
                  \end{eqnarray*}
                  \begin{block}{Note}
                      The assignment operator is right associated, so the production is $\text{expr}  \rightarrow  \text{expr-5 \textbf{=} expr}$ instead of $\text{expr}  \rightarrow  \text{expr \textbf{=} expr-5}$.
                  \end{block}
        \end{enumerate}
    }

\end{frame}

\subsection{Operations on Structures}

\begin{frame}
    \frametitle{Problems with operations on structures}

    \begin{itemize}
        \item Assignment to structures\\
              \lstinline{struct S s = t;}
        \item Structures as parameters\\
              \lstinline{void func(struct S)\{...\};}
        \item Structures as return values\\
              \lstinline{struct S func()\{...\};}
    \end{itemize}

\end{frame}

\begin{frame}[fragile]
    \frametitle{Assignment to structures}

    \begin{columns}
        \begin{column}[]{0.5\textwidth}
            \begin{itemize}
                \item Source code
                      \begin{lstlisting}[language=C]
struct S
{
    int member1;
    float member2;
};

s = t;
                    \end{lstlisting}
            \end{itemize}

        \end{column}

        \begin{column}[]{0.5\textwidth}
            \begin{itemize}
                \item Transformed code
                      \begin{lstlisting}[language=C]
struct S{
    int member1;
    float member2;
};

s.member1 = t.member1;
s.member2 = t.member2;
                \end{lstlisting}
            \end{itemize}
        \end{column}
    \end{columns}

\end{frame}

\begin{frame}[fragile]
    \frametitle{Structures as parameters}

    \begin{columns}
        \begin{column}[]{0.4\textwidth}
            \begin{itemize}
                \item Source code
                      \begin{lstlisting}[language=C]
void func(struct S s)
{
    ...
}

int main()
{
    func(t);
}
                    \end{lstlisting}
            \end{itemize}

        \end{column}

        \begin{column}[]{0.5\textwidth}
            \begin{itemize}
                \item Transformed code
                      \begin{lstlisting}[language=C]
void func(struct S *ptr)
{
    struct S s = *ptr;
}

int main()
{
    func(&t);
}
                \end{lstlisting}
            \end{itemize}
        \end{column}
    \end{columns}

\end{frame}

\begin{frame}[fragile]
    \frametitle{Structures as return values}

    \begin{columns}
        \begin{column}[]{0.4\textwidth}
            \begin{itemize}
                \item Source code
                      \begin{lstlisting}[language=C]
struct S func()
{
    ...
    return s;
}

int main()
{
    func();
}
                    \end{lstlisting}
            \end{itemize}

        \end{column}

        \begin{column}[]{0.7\textwidth}
            \begin{itemize}
                \item Transformed code
                      \begin{lstlisting}[language=C]
void func(struct S* ret_ptr)
{
    ...
    *ret_ptr = ret_value;
}

int main() {
    struct S ret_value;
    func(&ret_value);
}
                \end{lstlisting}
            \end{itemize}
        \end{column}
    \end{columns}

\end{frame}

\section{Classes and Templates}

\subsection{Non-Static Member Functions}

\begin{frame}[fragile]
    \frametitle{Functions declared within a structure}

    \begin{example}
        \small
        \begin{columns}
            \begin{column}[]{0.4\textwidth}
                \begin{lstlisting}[language=C]
struct S
{
    ...
    void func(){}
};
                \end{lstlisting}

            \end{column}

            \begin{column}[]{0.4\textwidth}
                \begin{lstlisting}[language=C]
int main()
{
    struct S s;
    s.func();
}
                \end{lstlisting}
            \end{column}
        \end{columns}
    \end{example}
\end{frame}

\begin{frame}[fragile]
    \frametitle{Transformation of member functions}

    \begin{columns}
        \begin{column}[]{0.4\textwidth}
            \begin{itemize}
                \item Source code
                      \begin{lstlisting}[language=C]
struct S
{
    ...
    void func(){}
};

int main() 
{
    struct S s;
    s.func();
}
                    \end{lstlisting}
            \end{itemize}
        \end{column}

        \begin{column}[]{0.7\textwidth}
            \begin{itemize}
                \item Transformed code
                      \begin{lstlisting}[language=C]
struct S
{
    ...
};
void S::func(struct S* this){}

int main()
{
    struct S s;
    S::func(&s);
}
                \end{lstlisting}
            \end{itemize}
        \end{column}
    \end{columns}

\end{frame}

\subsection{Operator Overloading}

\begin{frame}[fragile]
    \frametitle{Overload an operator}

    \begin{lstlisting}[language=C]
struct S {
    struct S operator+(struct S rhs)
    {...}
};

int main()
{
    struct S s;
    struct S t;
    ...
    s + t; // S::operator+ is called
}
    \end{lstlisting}
\end{frame}

\begin{frame}[fragile]
    \frametitle{Transform the overloading function}
    \begin{columns}
        \begin{column}[]{0.4\textwidth}
            \begin{itemize}
                \item Source code
                      \small
                      \begin{lstlisting}[language=C]
struct S
{
    S operator+(S)
    {...}
};



int main() {
    struct S s;
    struct S t;
    ...
    s + t;
}
                    \end{lstlisting}
            \end{itemize}

        \end{column}

        \begin{column}[]{0.8\textwidth}
            \begin{itemize}
                \item Transformed code
                      \small
                      \begin{lstlisting}[language=C]
void S::operator+(
    struct S* ret_ptr, 
    struct S* this
    struct S* rhs_ptr){
    struct S rhs = *rhs_ptr;
    ...
    *ret_ptr = ret_value;
}
int main(){
    struct S s; struct S t;
    ...
    struct S ret_value;
    S::operator+(&ret_value, &s, &t);
}
                \end{lstlisting}
            \end{itemize}
        \end{column}
    \end{columns}
\end{frame}

% \begin{frame}
%     \frametitle{Some details about my implementation}

%     \begin{block}{Function tables for operators}
%         The supported operators are arithmatic operators \lstinline{+,-,*,/}. \\
%         A function table for each of these operators is maintained, and is looked up every time the operation is performed.
%     \end{block}

%     \begin{block}{\textbf{Not} operator \textit{Overloading}}
%         Function overloading is not implemented. Once the operator is overriden, the overriding function will be invoked every time the operation is performed.
%     \end{block}

% \end{frame}
\subsection{Class Templates}

\begin{frame}[fragile]
    \frametitle{Generic programing}

    \begin{example}
        \footnotesize
        \begin{columns}
            \begin{column}[]{0.4\textwidth}
                \begin{lstlisting}[language=C++]
template<typename T>
struct stack{
    T top();
    void push(T);
};
                \end{lstlisting}

            \end{column}

            \begin{column}[]{0.4\textwidth}
                \begin{lstlisting}[language=C]
int main()
{
    stack<int> a;
    stack<string> b;
    a.push(42);
    b.push("42");
}
\end{lstlisting}
            \end{column}
        \end{columns}
    \end{example}
\end{frame}

\begin{frame}
    \frametitle{CFG for class templates}

    \begin{eqnarray*}
        \text{template-definition} & \rightarrow & \textbf{template} \textbf{ < typename ID, ...>} \\
        &  & \text{struct-definition}\\
        \\
        \text{type-specifier} & \rightarrow & \cdots \\
        & | & \textbf{typename ID} \\
        & | & \textbf{struct ID <}\text{declaration,...}\textbf{>}
    \end{eqnarray*}

    \begin{block}{Note}
        \small
        Due to the limitation of CFG, in the specification of a class template, the keyword \textit{typename} is required before a template parameter.
    \end{block}

\end{frame}

\begin{frame}[fragile]
    \frametitle{Translation process of a class template}

    \begin{columns}
        \begin{column}[]{0.5\textwidth}
            \small
            \begin{lstlisting}[language=C++]
template<typename T>
struct S{
    typename T a;
};

int main(){
    struct S<int> x;
}
                \end{lstlisting}
        \end{column}

        \begin{column}[]{0.5\textwidth}
            \Tree [.program [.{template\ \textbf{S}} [.{\lstinline{struct S}}
                        [.\lstinline{T t}
                            {T}
                            {a}
                        ]
                ]]
            [.{\lstinline{int main()\{...\}}}
            [.{\lstinline{struct S<int> x;}}
                {\lstinline{struct S<int>}}
                {x}
            ]
            ]
            ]

        \end{column}
    \end{columns}

\end{frame}

\begin{frame}
    \frametitle{Translation process of a class template}

    \begin{columns}
        \begin{column}[]{0.5\textwidth}
            \Tree [.{\alert<1>{program}}
            [.{\alert<2>{template S}}
                [.{\alert<6>{\lstinline{struct S}}}
                        [.{\alert<7>{\lstinline{T a}}}
                                {\alert<8>{T}}
                                {\alert<9>{a}}
                        ]
                ]]
            [.{\alert<3>{\lstinline{int main()\{...\}}}}
            [.{\alert<4>{\lstinline{struct S<int> x;}}}
                {\alert<5,10>{\lstinline{struct S<int>}}}
                {\alert<11>{x}}
            ]
            ]
            ]
        \end{column}

        \begin{column}[]{0.5\textwidth}
            \begin{itemize}
                \small
                \pause
                \item \textbf<2>{Record the pointer of node \lstinline{template S} in AST, do not analyse subtrees. }\pause
                      \pause
                      \pause
                \item \textbf<5>{\lstinline{S<int>} does not exist. Map parameter T to \lstinline{int}, analyse class template}\pause
                \item \textbf<6>{Construct structure \lstinline{S<int>}}\pause
                      \pause
                \item \textbf<8>{Use mapping to replace \lstinline{T} with \lstinline{int}}\pause
                      \pause
                \item \textbf<10>{\lstinline{S<int>} now is a complete type }\pause
                \item \textbf<11>{Declare a variable of type \lstinline{S<int>}}
            \end{itemize}
        \end{column}
    \end{columns}

\end{frame}

\end{document}